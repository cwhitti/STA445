% Options for packages loaded elsewhere
\PassOptionsToPackage{unicode}{hyperref}
\PassOptionsToPackage{hyphens}{url}
%
\documentclass[
]{article}
\usepackage{amsmath,amssymb}
\usepackage{iftex}
\ifPDFTeX
  \usepackage[T1]{fontenc}
  \usepackage[utf8]{inputenc}
  \usepackage{textcomp} % provide euro and other symbols
\else % if luatex or xetex
  \usepackage{unicode-math} % this also loads fontspec
  \defaultfontfeatures{Scale=MatchLowercase}
  \defaultfontfeatures[\rmfamily]{Ligatures=TeX,Scale=1}
\fi
\usepackage{lmodern}
\ifPDFTeX\else
  % xetex/luatex font selection
\fi
% Use upquote if available, for straight quotes in verbatim environments
\IfFileExists{upquote.sty}{\usepackage{upquote}}{}
\IfFileExists{microtype.sty}{% use microtype if available
  \usepackage[]{microtype}
  \UseMicrotypeSet[protrusion]{basicmath} % disable protrusion for tt fonts
}{}
\makeatletter
\@ifundefined{KOMAClassName}{% if non-KOMA class
  \IfFileExists{parskip.sty}{%
    \usepackage{parskip}
  }{% else
    \setlength{\parindent}{0pt}
    \setlength{\parskip}{6pt plus 2pt minus 1pt}}
}{% if KOMA class
  \KOMAoptions{parskip=half}}
\makeatother
\usepackage{xcolor}
\usepackage[margin=1in]{geometry}
\usepackage{color}
\usepackage{fancyvrb}
\newcommand{\VerbBar}{|}
\newcommand{\VERB}{\Verb[commandchars=\\\{\}]}
\DefineVerbatimEnvironment{Highlighting}{Verbatim}{commandchars=\\\{\}}
% Add ',fontsize=\small' for more characters per line
\usepackage{framed}
\definecolor{shadecolor}{RGB}{248,248,248}
\newenvironment{Shaded}{\begin{snugshade}}{\end{snugshade}}
\newcommand{\AlertTok}[1]{\textcolor[rgb]{0.94,0.16,0.16}{#1}}
\newcommand{\AnnotationTok}[1]{\textcolor[rgb]{0.56,0.35,0.01}{\textbf{\textit{#1}}}}
\newcommand{\AttributeTok}[1]{\textcolor[rgb]{0.13,0.29,0.53}{#1}}
\newcommand{\BaseNTok}[1]{\textcolor[rgb]{0.00,0.00,0.81}{#1}}
\newcommand{\BuiltInTok}[1]{#1}
\newcommand{\CharTok}[1]{\textcolor[rgb]{0.31,0.60,0.02}{#1}}
\newcommand{\CommentTok}[1]{\textcolor[rgb]{0.56,0.35,0.01}{\textit{#1}}}
\newcommand{\CommentVarTok}[1]{\textcolor[rgb]{0.56,0.35,0.01}{\textbf{\textit{#1}}}}
\newcommand{\ConstantTok}[1]{\textcolor[rgb]{0.56,0.35,0.01}{#1}}
\newcommand{\ControlFlowTok}[1]{\textcolor[rgb]{0.13,0.29,0.53}{\textbf{#1}}}
\newcommand{\DataTypeTok}[1]{\textcolor[rgb]{0.13,0.29,0.53}{#1}}
\newcommand{\DecValTok}[1]{\textcolor[rgb]{0.00,0.00,0.81}{#1}}
\newcommand{\DocumentationTok}[1]{\textcolor[rgb]{0.56,0.35,0.01}{\textbf{\textit{#1}}}}
\newcommand{\ErrorTok}[1]{\textcolor[rgb]{0.64,0.00,0.00}{\textbf{#1}}}
\newcommand{\ExtensionTok}[1]{#1}
\newcommand{\FloatTok}[1]{\textcolor[rgb]{0.00,0.00,0.81}{#1}}
\newcommand{\FunctionTok}[1]{\textcolor[rgb]{0.13,0.29,0.53}{\textbf{#1}}}
\newcommand{\ImportTok}[1]{#1}
\newcommand{\InformationTok}[1]{\textcolor[rgb]{0.56,0.35,0.01}{\textbf{\textit{#1}}}}
\newcommand{\KeywordTok}[1]{\textcolor[rgb]{0.13,0.29,0.53}{\textbf{#1}}}
\newcommand{\NormalTok}[1]{#1}
\newcommand{\OperatorTok}[1]{\textcolor[rgb]{0.81,0.36,0.00}{\textbf{#1}}}
\newcommand{\OtherTok}[1]{\textcolor[rgb]{0.56,0.35,0.01}{#1}}
\newcommand{\PreprocessorTok}[1]{\textcolor[rgb]{0.56,0.35,0.01}{\textit{#1}}}
\newcommand{\RegionMarkerTok}[1]{#1}
\newcommand{\SpecialCharTok}[1]{\textcolor[rgb]{0.81,0.36,0.00}{\textbf{#1}}}
\newcommand{\SpecialStringTok}[1]{\textcolor[rgb]{0.31,0.60,0.02}{#1}}
\newcommand{\StringTok}[1]{\textcolor[rgb]{0.31,0.60,0.02}{#1}}
\newcommand{\VariableTok}[1]{\textcolor[rgb]{0.00,0.00,0.00}{#1}}
\newcommand{\VerbatimStringTok}[1]{\textcolor[rgb]{0.31,0.60,0.02}{#1}}
\newcommand{\WarningTok}[1]{\textcolor[rgb]{0.56,0.35,0.01}{\textbf{\textit{#1}}}}
\usepackage{graphicx}
\makeatletter
\def\maxwidth{\ifdim\Gin@nat@width>\linewidth\linewidth\else\Gin@nat@width\fi}
\def\maxheight{\ifdim\Gin@nat@height>\textheight\textheight\else\Gin@nat@height\fi}
\makeatother
% Scale images if necessary, so that they will not overflow the page
% margins by default, and it is still possible to overwrite the defaults
% using explicit options in \includegraphics[width, height, ...]{}
\setkeys{Gin}{width=\maxwidth,height=\maxheight,keepaspectratio}
% Set default figure placement to htbp
\makeatletter
\def\fps@figure{htbp}
\makeatother
\setlength{\emergencystretch}{3em} % prevent overfull lines
\providecommand{\tightlist}{%
  \setlength{\itemsep}{0pt}\setlength{\parskip}{0pt}}
\setcounter{secnumdepth}{-\maxdimen} % remove section numbering
\ifLuaTeX
  \usepackage{selnolig}  % disable illegal ligatures
\fi
\IfFileExists{bookmark.sty}{\usepackage{bookmark}}{\usepackage{hyperref}}
\IfFileExists{xurl.sty}{\usepackage{xurl}}{} % add URL line breaks if available
\urlstyle{same}
\hypersetup{
  pdftitle={homework4},
  pdfauthor={Claire Whittington},
  hidelinks,
  pdfcreator={LaTeX via pandoc}}

\title{homework4}
\author{Claire Whittington}
\date{2023-10-16}

\begin{document}
\maketitle

\hypertarget{exercise-1-7-pts.-in-exercise-1}{%
\subsection{1) Exercise 1 -- 7 pts. In exercise
1:}\label{exercise-1-7-pts.-in-exercise-1}}

• Complete any 7 of the 9 scenarios from a -- i. Ensure the scenarios
are clearly identifiable. • The string \textless- c() at the start of
each chunk is where you will enter several strings to show that the 2
lines of code underneath do what you say they are doing. • Your test
string will need to include several expressions return a TRUE result and
a FALSE result from the code. For example, if you state the code checks
for the letter ``e'', your test string needs to have some expressions
with the letter e , to return TRUE result, and some expressions without
the letter e, to show the code returns FALSE when no ``e'' is found. •
If you need help coming up with good test strings (particularly as the
scenarios get more complex), please ask.

CHEATSHEET

=====================================

Character Types Interpretation

\begin{itemize}
\item
  abc Letters abc exactly
\item
  123 Digits 123 exactly
\item
  \d    Any Digit
\item
  \D    Any Non-digit character
\item
  \w    Any Alphanumeric character
\item
  \W    Any Non-alphanumeric character
\item
  \s    Any White space
\item
  \S    Any Non-white space character
\item
  . Any Character (The wildcard!)
\item
  \^{} Beginning of input string
\item
  \$ End of input string
\end{itemize}

Grouping Interpretation

\begin{itemize}
\item
  {[}abc{]} Only a, b, or c
\item
  {[}\^{}abc{]} Not a, b, nor c
\item
  {[}a-z{]} Characters a to z
\item
  {[}A-Z{]} Characters A to Z
\item
  {[}0-9{]} Numbers 0 to 9
\item
  {[}a-zA-Z{]} Characters a to z or A to Z
\item
  () Capture Group
\item
  (a(bc)) Capture Sub-group
\item
  (abc\textbar def) Matches abc or def
\end{itemize}

Group Modifiers Interpretation

\begin{itemize}
\item
  \begin{itemize}
  \tightlist
  \item
    Zero or more repetitions of previous (greedy)
  \end{itemize}
\item
  \begin{itemize}
  \tightlist
  \item
    One or more repetitions of previous (greedy)
  \end{itemize}
\item
  ? Previous group is optional
\item
  \{m\} m repetitions of the previous
\item
  \{m,n\} Between m and n repetitions of the previous
\item
  *? Zero or more repetitions of previous (not-greedy).

  Obnoxiously the ? is modifying the modifier here and so has a
  different interpretation than when modifying a group.
\item
  +? One or more repetitions of previous (not-greedy)
\end{itemize}

=====================================

\begin{enumerate}
\def\labelenumi{\alph{enumi})}
\tightlist
\item
  This regular expression scans ``strings'' variable and creates a
  separate column which returns either F or F if there is an `a' in the
  scanned string
\end{enumerate}

\begin{Shaded}
\begin{Highlighting}[]
\NormalTok{strings }\OtherTok{\textless{}{-}} \FunctionTok{c}\NormalTok{(}\StringTok{"January"}\NormalTok{,}\StringTok{"February"}\NormalTok{,}\StringTok{"March"}\NormalTok{,}\StringTok{"April"}\NormalTok{,}
            \StringTok{"May"}\NormalTok{,}\StringTok{"June"}\NormalTok{,}\StringTok{"July"}\NormalTok{,}\StringTok{"August"}\NormalTok{,}
            \StringTok{"September"}\NormalTok{,}\StringTok{"October"}\NormalTok{,}\StringTok{"November"}\NormalTok{)}

\FunctionTok{data.frame}\NormalTok{( }\AttributeTok{string =}\NormalTok{ strings ) }\SpecialCharTok{\%\textgreater{}\%}
  \FunctionTok{mutate}\NormalTok{( }\AttributeTok{result =} \FunctionTok{str\_detect}\NormalTok{(string, }\StringTok{\textquotesingle{}a\textquotesingle{}}\NormalTok{) )}
\end{Highlighting}
\end{Shaded}

\begin{verbatim}
##       string result
## 1    January   TRUE
## 2   February   TRUE
## 3      March   TRUE
## 4      April  FALSE
## 5        May   TRUE
## 6       June  FALSE
## 7       July  FALSE
## 8     August  FALSE
## 9  September  FALSE
## 10   October  FALSE
## 11  November  FALSE
\end{verbatim}

\begin{enumerate}
\def\labelenumi{\alph{enumi})}
\setcounter{enumi}{1}
\tightlist
\item
  This regular expression evaluates each string in a vector of strings
  for the string `ab'
\end{enumerate}

\begin{Shaded}
\begin{Highlighting}[]
\CommentTok{\# This regular expression matches:  Insert your answer here...}
\NormalTok{strings }\OtherTok{\textless{}{-}} \FunctionTok{c}\NormalTok{(}\StringTok{"Ryan"}\NormalTok{,}\StringTok{"Abby"}\NormalTok{,}\StringTok{"Gabriel"}\NormalTok{,}\StringTok{"Ben"}\NormalTok{,}\StringTok{"Olivia"}\NormalTok{)}
\FunctionTok{data.frame}\NormalTok{( }\AttributeTok{string =}\NormalTok{ strings ) }\SpecialCharTok{\%\textgreater{}\%}
  \FunctionTok{mutate}\NormalTok{( }\AttributeTok{result =} \FunctionTok{str\_detect}\NormalTok{(string, }\StringTok{\textquotesingle{}ab\textquotesingle{}}\NormalTok{) )}
\end{Highlighting}
\end{Shaded}

\begin{verbatim}
##    string result
## 1    Ryan  FALSE
## 2    Abby  FALSE
## 3 Gabriel   TRUE
## 4     Ben  FALSE
## 5  Olivia  FALSE
\end{verbatim}

\begin{enumerate}
\def\labelenumi{\alph{enumi})}
\setcounter{enumi}{2}
\tightlist
\item
  This regular expression evaluates each string in a vector of strings
  for the letters a OR b. Returns TRUE if a OR b is detected in a string
\end{enumerate}

\begin{Shaded}
\begin{Highlighting}[]
\NormalTok{strings }\OtherTok{\textless{}{-}} \FunctionTok{c}\NormalTok{(}\StringTok{"boat"}\NormalTok{,}\StringTok{"moat"}\NormalTok{,}\StringTok{"goat"}\NormalTok{,}\StringTok{"woke"}\NormalTok{)}
\FunctionTok{data.frame}\NormalTok{( }\AttributeTok{string =}\NormalTok{ strings ) }\SpecialCharTok{\%\textgreater{}\%}
  \FunctionTok{mutate}\NormalTok{( }\AttributeTok{result =} \FunctionTok{str\_detect}\NormalTok{(string, }\StringTok{\textquotesingle{}[ab]\textquotesingle{}}\NormalTok{) )}
\end{Highlighting}
\end{Shaded}

\begin{verbatim}
##   string result
## 1   boat   TRUE
## 2   moat   TRUE
## 3   goat   TRUE
## 4   woke  FALSE
\end{verbatim}

\begin{enumerate}
\def\labelenumi{\alph{enumi})}
\setcounter{enumi}{3}
\tightlist
\item
  This regular expression evaluates each string in a vector of strings
  for the letters a AND b. Returns TRUE if a AND b is detected in a
  string
\end{enumerate}

\begin{Shaded}
\begin{Highlighting}[]
\NormalTok{strings }\OtherTok{\textless{}{-}} \FunctionTok{c}\NormalTok{(}\StringTok{"boat"}\NormalTok{,}\StringTok{"moat"}\NormalTok{,}\StringTok{"goat"}\NormalTok{,}\StringTok{"woke"}\NormalTok{)}
\FunctionTok{data.frame}\NormalTok{( }\AttributeTok{string =}\NormalTok{ strings ) }\SpecialCharTok{\%\textgreater{}\%}
  \FunctionTok{mutate}\NormalTok{( }\AttributeTok{result =} \FunctionTok{str\_detect}\NormalTok{(string, }\StringTok{\textquotesingle{}\^{}[ab]\textquotesingle{}}\NormalTok{) )}
\end{Highlighting}
\end{Shaded}

\begin{verbatim}
##   string result
## 1   boat   TRUE
## 2   moat  FALSE
## 3   goat  FALSE
## 4   woke  FALSE
\end{verbatim}

\begin{enumerate}
\def\labelenumi{\alph{enumi})}
\setcounter{enumi}{4}
\tightlist
\item
  This regular expression evaluates each string in a vector of strings
  for one or more digits (`//d') followed by a space (`//s') and either
  `a' or `A' (`{[}aA{]}')
\end{enumerate}

\begin{Shaded}
\begin{Highlighting}[]
\NormalTok{strings }\OtherTok{\textless{}{-}} \FunctionTok{c}\NormalTok{(}\StringTok{"123 Apple"}\NormalTok{, }\StringTok{"456 a"}\NormalTok{, }\StringTok{"321Airplane"}\NormalTok{, }\StringTok{"789 B"}\NormalTok{, }\StringTok{"No numbers here"}\NormalTok{)}
\FunctionTok{data.frame}\NormalTok{( }\AttributeTok{string =}\NormalTok{ strings ) }\SpecialCharTok{\%\textgreater{}\%}
  \FunctionTok{mutate}\NormalTok{( }\AttributeTok{result =} \FunctionTok{str\_detect}\NormalTok{(string, }\StringTok{\textquotesingle{}}\SpecialCharTok{\textbackslash{}\textbackslash{}}\StringTok{d+}\SpecialCharTok{\textbackslash{}\textbackslash{}}\StringTok{s[aA]\textquotesingle{}}\NormalTok{) )}
\end{Highlighting}
\end{Shaded}

\begin{verbatim}
##            string result
## 1       123 Apple   TRUE
## 2           456 a   TRUE
## 3     321Airplane  FALSE
## 4           789 B  FALSE
## 5 No numbers here  FALSE
\end{verbatim}

\begin{enumerate}
\def\labelenumi{\alph{enumi})}
\setcounter{enumi}{5}
\tightlist
\item
  This regular expression evaluates each string for one or more digits
  (`//d') followed by zero or more whitespace characters (`//s*') and
  then either `a' or `A' (`{[}aA{]}')
\end{enumerate}

\begin{Shaded}
\begin{Highlighting}[]
\NormalTok{strings }\OtherTok{\textless{}{-}} \FunctionTok{c}\NormalTok{(}\StringTok{"123 Apple"}\NormalTok{, }\StringTok{"456 a"}\NormalTok{, }\StringTok{"321Airplane"}\NormalTok{, }\StringTok{"789 B"}\NormalTok{, }\StringTok{"No numbers here"}\NormalTok{)}
\FunctionTok{data.frame}\NormalTok{( }\AttributeTok{string =}\NormalTok{ strings ) }\SpecialCharTok{\%\textgreater{}\%}
  \FunctionTok{mutate}\NormalTok{( }\AttributeTok{result =} \FunctionTok{str\_detect}\NormalTok{(string, }\StringTok{\textquotesingle{}}\SpecialCharTok{\textbackslash{}\textbackslash{}}\StringTok{d+}\SpecialCharTok{\textbackslash{}\textbackslash{}}\StringTok{s*[aA]\textquotesingle{}}\NormalTok{) )}
\end{Highlighting}
\end{Shaded}

\begin{verbatim}
##            string result
## 1       123 Apple   TRUE
## 2           456 a   TRUE
## 3     321Airplane   TRUE
## 4           789 B  FALSE
## 5 No numbers here  FALSE
\end{verbatim}

\begin{enumerate}
\def\labelenumi{\alph{enumi})}
\setcounter{enumi}{6}
\tightlist
\item
  This regular expression evaluates each string for Zero or more
  repetitions of periods in a string
\end{enumerate}

\begin{Shaded}
\begin{Highlighting}[]
\NormalTok{strings }\OtherTok{\textless{}{-}} \FunctionTok{c}\NormalTok{(}\StringTok{"abc"}\NormalTok{,}\StringTok{"abc."}\NormalTok{,}\StringTok{"abc.."}\NormalTok{, }\StringTok{"abc!!!"}\NormalTok{, }\StringTok{""}\NormalTok{, }\ConstantTok{NA}\NormalTok{)}
\FunctionTok{data.frame}\NormalTok{( }\AttributeTok{string =}\NormalTok{ strings ) }\SpecialCharTok{\%\textgreater{}\%}
  \FunctionTok{mutate}\NormalTok{( }\AttributeTok{result =} \FunctionTok{str\_detect}\NormalTok{(string, }\StringTok{\textquotesingle{}.*\textquotesingle{}}\NormalTok{) )}
\end{Highlighting}
\end{Shaded}

\begin{verbatim}
##   string result
## 1    abc   TRUE
## 2   abc.   TRUE
## 3  abc..   TRUE
## 4 abc!!!   TRUE
## 5          TRUE
## 6   <NA>     NA
\end{verbatim}

\hypertarget{exercise-2-3-pts}{%
\subsection{(2) Exercise 2 -- 3 pts}\label{exercise-2-3-pts}}

The following file names were used in a camera trap study. The S number
represents the site, P is the plot within a site, C is the camera number
within the plot, the first string of numbers is the YearMonthDay and the
second string of numbers is the HourMinuteSecond.

\begin{Shaded}
\begin{Highlighting}[]
\NormalTok{file.names }\OtherTok{\textless{}{-}} \FunctionTok{data.frame}\NormalTok{( }\StringTok{\textquotesingle{}S123.P2.C10\_20120621\_213422.jpg\textquotesingle{}}\NormalTok{,}
                 \StringTok{\textquotesingle{}S10.P1.C1\_20120622\_050148.jpg\textquotesingle{}}\NormalTok{,}
                 \StringTok{\textquotesingle{}S187.P2.C2\_20120702\_023501.jpg\textquotesingle{}}\NormalTok{)}
\end{Highlighting}
\end{Shaded}

Produce a data frame with columns corresponding to the site, plot,
camera, year, month, day, hour, minute, and second for these three file
names. So we want to produce code that will create the data frame:

\begin{Shaded}
\begin{Highlighting}[]
\CommentTok{\# Split the file names using a regular expression}
\NormalTok{split\_data }\OtherTok{\textless{}{-}} \FunctionTok{data.frame}\NormalTok{( }\FunctionTok{str\_split\_fixed}\NormalTok{( file.names, }\StringTok{"}\SpecialCharTok{\textbackslash{}\textbackslash{}}\StringTok{.|\_"}\NormalTok{, }\AttributeTok{n=}\DecValTok{6}\NormalTok{) )}

\CommentTok{\#data.frame( string = strings ) \%\textgreater{}\%}
\CommentTok{\#  mutate( result = str\_detect(string, \textquotesingle{}ab\textquotesingle{}) )}


\FunctionTok{colnames}\NormalTok{(split\_data) }\OtherTok{\textless{}{-}} \FunctionTok{c}\NormalTok{(}\StringTok{"Site"}\NormalTok{, }\StringTok{"Plot"}\NormalTok{, }\StringTok{"Camera"}\NormalTok{, }\StringTok{"Date"}\NormalTok{, }\StringTok{"Time"}\NormalTok{, }\StringTok{"Format"}\NormalTok{)}

\NormalTok{split\_data}
\end{Highlighting}
\end{Shaded}

\begin{verbatim}
##   Site Plot Camera     Date   Time Format
## 1 S123   P2    C10 20120621 213422    jpg
## 2  S10   P1     C1 20120622 050148    jpg
## 3 S187   P2     C2 20120702 023501    jpg
\end{verbatim}

\begin{Shaded}
\begin{Highlighting}[]
\NormalTok{split\_data }\SpecialCharTok{\%\textgreater{}\%} \FunctionTok{mutate}\NormalTok{(}
  \AttributeTok{Year =} \FunctionTok{str\_sub}\NormalTok{(Date,}\DecValTok{1}\NormalTok{,}\DecValTok{4}\NormalTok{),}
  \AttributeTok{Month =} \FunctionTok{str\_sub}\NormalTok{(Date,}\DecValTok{5}\NormalTok{,}\DecValTok{6}\NormalTok{),}
  \AttributeTok{Day =} \FunctionTok{str\_sub}\NormalTok{(Date,}\DecValTok{7}\NormalTok{,}\DecValTok{8}\NormalTok{),}
  \AttributeTok{Hour =} \FunctionTok{str\_sub}\NormalTok{(Time,}\DecValTok{1}\NormalTok{,}\DecValTok{2}\NormalTok{),}
  \AttributeTok{Minute =} \FunctionTok{str\_sub}\NormalTok{(Time,}\DecValTok{3}\NormalTok{,}\DecValTok{4}\NormalTok{),}
  \AttributeTok{Second =} \FunctionTok{str\_sub}\NormalTok{(Time,}\DecValTok{5}\NormalTok{,}\DecValTok{6}\NormalTok{),}
\NormalTok{)}
\end{Highlighting}
\end{Shaded}

\begin{verbatim}
##   Site Plot Camera     Date   Time Format Year Month Day Hour Minute Second
## 1 S123   P2    C10 20120621 213422    jpg 2012    06  21   21     34     22
## 2  S10   P1     C1 20120622 050148    jpg 2012    06  22   05     01     48
## 3 S187   P2     C2 20120702 023501    jpg 2012    07  02   02     35     01
\end{verbatim}

\begin{enumerate}
\def\labelenumi{(\arabic{enumi})}
\setcounter{enumi}{2}
\tightlist
\item
  Exercise 3 -- 3 pts The full text from Lincoln's Gettysburg Address is
  given below. Calculate the mean word length
\end{enumerate}

\begin{Shaded}
\begin{Highlighting}[]
\NormalTok{Gettysburg }\OtherTok{\textless{}{-}} \StringTok{\textquotesingle{}Four score and seven years ago our fathers brought forth on this }
\StringTok{continent, a new nation, conceived in Liberty, and dedicated to the proposition }
\StringTok{that all men are created equal.}

\StringTok{Now we are engaged in a great civil war, testing whether that nation, or any }
\StringTok{nation so conceived and so dedicated, can long endure. We are met on a great }
\StringTok{battle{-}field of that war. We have come to dedicate a portion of that field, as }
\StringTok{a final resting place for those who here gave their lives that that nation might }
\StringTok{live. It is altogether fitting and proper that we should do this.}

\StringTok{But, in a larger sense, we can not dedicate {-}{-} we can not consecrate {-}{-} we can }
\StringTok{not hallow {-}{-} this ground. The brave men, living and dead, who struggled here, }
\StringTok{have consecrated it, far above our poor power to add or detract. The world will }
\StringTok{little note, nor long remember what we say here, but it can never forget what }
\StringTok{they did here. It is for us the living, rather, to be dedicated here to the }
\StringTok{unfinished work which they who fought here have thus far so nobly advanced. It }
\StringTok{is rather for us to be here dedicated to the great task remaining before us {-}{-} }
\StringTok{that from these honored dead we take increased devotion to that cause for which }
\StringTok{they gave the last full measure of devotion {-}{-} that we here highly resolve that }
\StringTok{these dead shall not have died in vain {-}{-} that this nation, under God, shall }
\StringTok{have a new birth of freedom {-}{-} and that government of the people, by the people, }
\StringTok{for the people, shall not perish from the earth.\textquotesingle{}}
\end{Highlighting}
\end{Shaded}

\begin{Shaded}
\begin{Highlighting}[]
\NormalTok{cleaned\_words\_1 }\OtherTok{\textless{}{-}} \FunctionTok{str\_split}\NormalTok{(Gettysburg, }\StringTok{"[\^{}a{-}zA{-}Z{-}]"}\NormalTok{   )}

\NormalTok{valid\_words }\OtherTok{=} \DecValTok{0}
\NormalTok{sum }\OtherTok{=} \DecValTok{0}

\ControlFlowTok{for}\NormalTok{ ( index }\ControlFlowTok{in} \DecValTok{1}\SpecialCharTok{:}\FunctionTok{length}\NormalTok{(cleaned\_words\_1[[}\DecValTok{1}\NormalTok{]]) )}
\NormalTok{\{}
\NormalTok{  word }\OtherTok{=}\NormalTok{ (cleaned\_words\_1[[}\DecValTok{1}\NormalTok{]][index])}
  
  \ControlFlowTok{if}\NormalTok{ ( ( }\FunctionTok{nchar}\NormalTok{(word) }\SpecialCharTok{\textgreater{}} \DecValTok{0}\NormalTok{) }\SpecialCharTok{\&\&}\NormalTok{ (word }\SpecialCharTok{!=} \StringTok{"{-}{-}"}\NormalTok{) )}
\NormalTok{  \{}
\NormalTok{    cleaned\_word }\OtherTok{\textless{}{-}} \FunctionTok{str\_replace}\NormalTok{(word, }\StringTok{"{-}"}\NormalTok{, }\StringTok{""}\NormalTok{)}
\NormalTok{    sum }\OtherTok{=}\NormalTok{ sum }\SpecialCharTok{+} \FunctionTok{nchar}\NormalTok{(cleaned\_word)}
\NormalTok{    valid\_words }\OtherTok{=}\NormalTok{ valid\_words }\SpecialCharTok{+} \DecValTok{1}
\NormalTok{  \}}
  
\NormalTok{\}}

\NormalTok{avg\_len }\OtherTok{=}\NormalTok{ sum }\SpecialCharTok{/}\NormalTok{ valid\_words}
\NormalTok{avg\_len}
\end{Highlighting}
\end{Shaded}

\begin{verbatim}
## [1] 4.239852
\end{verbatim}

\begin{enumerate}
\def\labelenumi{(\arabic{enumi})}
\setcounter{enumi}{3}
\tightlist
\item
  Turned in by the due date/time -- 2 points.
\end{enumerate}

\end{document}
